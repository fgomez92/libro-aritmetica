\chapter{Fundamentos}

En sus inicios, la necesidad de resolver problemas fue el motor m\'as importante que motiv\'o el desarrollo de la matem\'atica. No obstante, la idea de resolver problemas dista mucho del concepto actual; no eran ecuaciones, ni c\'alculos complejos. De hecho, dado que la matem\'atica es tan antigua como la civilizaci\'on misma, los primeros problemas a los que se enfrentaron los humanos primitivos fueron tan simples como contar la cantidad de frutas recolectadas.\\

\noindent Aunque el acto de contar puede parecer muy sencillo, es importante entender el contexto. Para contar fue necesario crear un nuevo concepto; el \emph{n\'umero}.  Dado que desde la infancia somos educados en un mundo lleno de n\'umeros, resulta casi imposible entender c\'omo \\

\noindent Despu\-es de esta breve introducci\-on ser\-ia prudente preguntarnos, ?`qu\'e es contar? Antes de responder esta pregunta, es necesario entender el contexto. Contar es una acci\'on asociada a colecciones de cosas, en donde la palabra cosa es usada en el contexto m\'as general posible. Es decir, cuando hablamos de 

\begin{tcolorbox}
\emph{Contar} consiste en asignarle un s\'imbolo, llamado n\'umero, a cada colecci\'on de cosas de forma que dos colecciones con la misma cardinalidad reciben el mismo n\'umero.
\end{tcolorbox}

\section{La suma}

\noindent .

\begin{ex*}
Hola
\end{ex*}